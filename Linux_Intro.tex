\documentclass{beamer}
\usetheme{Copenhagen}
\usepackage[utf8]{inputenc}


%\usepackage{graphicx}
%\usepackage{subfigure}
%\usepackage{multimedia}
\usepackage{times}  % fonts are up to you
\usepackage{graphics}
\usepackage{amsmath}
\usepackage{media9}
\usepackage{hyperref}
\usepackage{psfrag}
%\usepackage[style=authoryear]{biblatex}
%\bibliography{/Users/ali/Library/texmf/bibtex/bib/references}


\setbeamertemplate{bibliography item}[text]
%\usepackage[backend=bibtex, style=authoryear]{biblatex}
%\addbibresource{/Users/ali/Library/texmf/bibtex/bib/references.bib}
\newcommand{\customcite}[1]{\citeauthor{#1}, \citeyear{#1}}
\newcommand\smallFont{\fontsize{8}{7.2}\selectfont}   %Change font size.
\newcommand\mCite[1]{[\cite{#1}, \citetitle{#1}]}  %Prints name and title
\newcommand\FrameText[1]{
\begin{textblock*}{\paperwidth}(0pt,\textheight)
	\vspace{1.0cm}
    \raggedleft \smallFont #1
\end{textblock*}}

%Get rid of ugly copenhagen default symbol for enumerate
\setbeamertemplate{enumerate items}[default]   


% Create code text
% https://tex.stackexchange.com/questions/65291/code-snippet-in-text
\definecolor{codegray}{gray}{0.9}
\newcommand{\code}[1]{\colorbox{codegray}{\texttt{#1}}}




%Information to be included in the title page:
\title{Introduction to Linux}
\author{Ali Snedden}
\institute{Nationwide Children's Hospital}
\date{April 5, 2019}
 
 
 
\begin{document}
 
\frame{\titlepage}




\begin{frame}
\frametitle{How to Connect}
Windows:
\begin{itemize}
    \item Open PuTTY
    \item Window Session $\Rightarrow$ Host Name field : username@10.70.250.101
    \item Click ``Open" to log in.
    \item Enter password
\end{itemize}

Mac:
\begin{itemize}
    \item Open Terminal (Finder $\Rightarrow$ Utilities $\Rightarrow$ Terminal)
    \item \code{ssh -X username@10.70.250.101}
\end{itemize}

\end{frame}


\begin{frame}
\frametitle{Cluster Architecture}
\begin{picture}(320,250)  %must be related to where it is centered
%\put(-30, 200){\includegraphics[height=0.8in]{images/Panoramic_Austin.jpg}}
\put(0, 70){\includegraphics[height=2.5in]{images/GPFS_File.eps}}
\end{picture}
\end{frame}



\begin{frame}
\frametitle{What is Linux?}
Operating System
\begin{itemize}
    \item Bootloader 
    \item Kernel - interface between the hardware and operating system. Manages processes.
    \item Daemons - background processes
    \item Graphical Interface -  X11 or X
    \item Shell - a.k.a. ``The Command Line".  This will be where you interface with Linux.
\end{itemize}
\end{frame}
 

\begin{frame}
\frametitle{Why use Linux?}
\begin{itemize}
    \item Package managers (e.g. \code{yum})
    \bigskip
    \item Built for programming. Comes with compilers, editors etc.
    \bigskip
    \item Great online resources (e.g. https://stackoverflow.com/)
    \bigskip
    \item Customizable
    \bigskip
    \item Dominant HPC platform.
\end{itemize}
\end{frame}
 

\begin{frame}
\frametitle{Types of Users}
\begin{itemize}
    \item Regular users (you)
    \bigskip

    \item Priveleged users (Yuan and myself)
    \bigskip

    \begin{itemize}
        \item Can modify \code{/usr}, \code{/opt}, etc.
        \bigskip

        \item \code{sudo}
    \end{itemize}
\end{itemize}
\end{frame}


\begin{frame}
\frametitle{The ``Shell"}
%\begin{picture}(320,250)  %must be related to where it is centered
%%\put(-30, 200){\includegraphics[height=0.8in]{images/Panoramic_Austin.jpg}}
%\put(100, 100){\includegraphics[height=1.8in]{images/shell.eps}}
%\end{picture}
%\begin{tabular}{cl}  
%  \begin{tabular}{c}
%    \includegraphics[height=7cm, width=4.5cm]{images/shell.eps}
%    \end{tabular}
%    & \begin{tabular}{l}
%      \parbox{0.5\linewidth}{%  change the parbox width as appropiate
    \begin{itemize}
        \item Used at the command line, i.e. the `terminal'
        \bigskip
        \item The user interface with the operating system
        \bigskip
        \item Bash is the default shell on Baker.
    \end{itemize}
%    }
%    \end{tabular}  \\
%\end{tabular}
\end{frame}


\begin{frame}
\frametitle{Text Editors}
\begin{itemize}
    \item \code{vim}   : Command line based
    \bigskip
    \item \code{emacs} : Command line based
    \bigskip
    \item \code{gedit} : Graphical  based
\end{itemize}
\end{frame}

%\begin{frame}
%\begin{picture}(320,250)  %must be related to where it is centered
%\put(20, 10){\includegraphics[height=3.5in, width=3.5in]{images/vim.eps}}
%\end{picture}
%\frametitle{Vim}
%\end{frame}

\begin{frame}
\frametitle{Basic Commands}
\code{vim} - text editor
\bigskip
\begin{itemize}
    \item \code{vim $\sim$/Scatch/tmp.txt}
    \bigskip
    \item Command and Edit modes.
    \bigskip
    \item \code{esc} to enter command mode
    \bigskip
    \item \code{i} to enter edit mode
    \bigskip
    \item To save, in command mode : \code{esc} \code{:w}
    \bigskip
    \item To quit, in command mode : \code{esc} \code{:q}
\end{itemize}
\end{frame}

\begin{frame}
\frametitle{Basic Commands}
\code{ls} - List directories and files. E.g.
\bigskip
\begin{itemize}
    \item \code{ls /opt/python }
    \bigskip

    \item \code{ls -l /opt/python} 
\end{itemize}
\end{frame}



\begin{frame}
\frametitle{Basic Commands}
\code{mkdir} - Make directories. E.g.
\bigskip
\begin{itemize}
    \item \code{mkdir $\sim$/Scratch/newdir}
\end{itemize}
\end{frame}


\begin{frame}
\frametitle{Basic Commands}
\code{man} - prints user manual for command. E.g.
\bigskip
\begin{itemize}
    \item \code{man ls}
    \bigskip
    \item Use \code{d} and \code{b} to navigate forward and back.
    \bigskip
    \item Use \code{q} to quit
    \bigskip
    \item Use \code{/somestring} to search for \code{somestring} within the man page.
\end{itemize}
\end{frame}


\begin{frame}
\frametitle{Basic Commands}
\code{cd}- Change directory. E.g.
\bigskip
\begin{itemize}
    \item \code{cd $\sim$/Scratch}
    \bigskip
    \item \code{cd}
    \bigskip
    \item \code{cd ..}
    \bigskip
    \item \code{cd $\sim$} 
\end{itemize}
\end{frame}


\begin{frame}
\frametitle{Basic Commands}
\code{pwd}- print working directory. E.g.
\bigskip
\begin{itemize}
    \item \code{pwd}
\end{itemize}
\end{frame}


\begin{frame}
\frametitle{Basic Commands}
\code{touch}- Create empty file. E.g.
\bigskip
\begin{itemize}
    \item \code{touch $\sim$/Scratch/file.txt}
\end{itemize}
\end{frame}


\begin{frame}
\frametitle{Basic Commands}
\code{mv}- move file or rename file. E.g.
\bigskip
\begin{itemize}
    \item \code{mv $\sim$/Scratch/file.txt $\sim$/Scratch/file2.txt}
\end{itemize}
\end{frame}

\begin{frame}
\frametitle{Basic Commands}
\code{cp}- copy files or directories
\bigskip
\begin{itemize}
    \item \code{cp $\sim$/Scratch/file.txt $\sim$/Scratch/file2.txt}
    \bigskip
    \item \code{cp -r $\sim$/Scratch/newdir $\sim$/Scratch/newdir2}
\end{itemize}
\end{frame}


\begin{frame}
\frametitle{Basic Commands}
\code{rm}- delete files or directories. Be Careful. E.g.
\bigskip
\begin{itemize}
    \item \code{rm $\sim$/Scratch/file.txt}
    \bigskip
    \item \code{rm -r $\sim$/Scratch/newdir}
\end{itemize}
\end{frame}


\begin{frame}
\frametitle{Basic Commands}
\code{cat}- concatenate files. E.g.
\bigskip
\begin{itemize}
    \item \code{cat /opt/modulefiles/gcc-4.9.2}
\end{itemize}
\end{frame}


\begin{frame}
\frametitle{Basic Commands}
\code{echo}- echo back string. E.g.
\bigskip
\begin{itemize}
    \item \code{echo "Hello World"}
\end{itemize}
\end{frame}


\begin{frame}
\frametitle{Permissions}
\code{chmod}- Change permissions.
E.g.
\begin{itemize}
    \item \code{ls -l $\sim$/Scratch/file.txt}, yields : 
    \hspace{5mm} \code{-rw-r--r--  1 user group 0 Apr  5 09:01 file.txt}
    \smallskip
    \item \code{chmod 755 somescript.sh}
    \bigskip
    \item \code{ls -l $\sim$/Scratch/file.txt}, yields : 
    \hspace{5mm} \code{-rwxr-xr-x  1 user group 0 Apr  5 09:01 file.txt}
\end{itemize}
\bigskip

Explanation:
\begin{itemize}
    \item First number is you.
    \item Second number is for your group (i.e. lab).
    \item Third number is for everyone else.
    \item \code{r} = 4, \code{w} = 2, \code{x} = 1
    \item \code{7} = \code{4} + \code{2} + \code{1} = read (\code{r}) + write (\code{w}) +  execute(\code{x})
    \item \code{5} = \code{4} + \code{1} = read (\code{r}), execute(\code{x})
\end{itemize}
\end{frame}

\begin{frame}
\frametitle{Homework}
Create a bash script that says ``Hello World". Make it executable. 

\bigskip

\emph{HINT}, you'll need to add $\code{\#^^21/bin/bash}$ at the top of your script
\end{frame}

\end{document}





